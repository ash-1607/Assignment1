\let\negmedspace\undefined
\let\negthickspace\undefined
\documentclass[journal,12pt]{IEEEtran}

%\documentclass[conference]{IEEEtran}
%\IEEEoverridecommandlockouts
% The preceding line is only needed to identify funding in the first footnote. If that is unneeded, please comment it out.
\usepackage{cite}
\usepackage{amsmath,amssymb,amsfonts,amsthm}
\usepackage{algorithmic}
\usepackage{graphicx}
\usepackage{textcomp}
\usepackage{xcolor}
\usepackage{txfonts}
\usepackage{listings}
\usepackage{enumitem}
\usepackage{mathtools}
\usepackage{gensymb}
\usepackage[breaklinks=true]{hyperref}
\usepackage{tkz-euclide} % loads  TikZ and tkz-base
\usepackage{listings}
\usepackage{caption}
\newcommand{\Comb}[2]{{}^{#1}C_{#2}} % Define the \Comb command
\DeclareMathOperator*{\Res}{Res}
%\renewcommand{\baselinestretch}{2}
\renewcommand\thesection{\arabic{section}}
\renewcommand\thesubsection{\thesection.\arabic{subsection}}
\renewcommand\thesubsubsection{\thesubsection.\arabic{subsubsection}}

\renewcommand\thesectiondis{\arabic{section}}
\renewcommand\thesubsectiondis{\thesectiondis.\arabic{subsection}}
\renewcommand\thesubsubsectiondis{\thesubsectiondis.\arabic{subsubsection}}

% correct bad hyphenation here
\hyphenation{op-tical net-works semi-conduc-tor}
\def\inputGnumericTable{}                                 %%

\lstset{
%language=C,
frame=single, 
breaklines=true,
columns=fullflexible
}

\begin{document}
%


\newtheorem{theorem}{Theorem}[section]
\newtheorem{problem}{Problem}
\newtheorem{proposition}{Proposition}[section]
\newtheorem{lemma}{Lemma}[section]
\newtheorem{corollary}[theorem]{Corollary}
\newtheorem{example}{Example}[section]
\newtheorem{definition}[problem]{Definition}

\newcommand{\BEQA}{\begin{eqnarray}}
\newcommand{\EEQA}{\end{eqnarray}}
\newcommand{\define}{\stackrel{\triangle}{=}}
\newcommand\tab[1][1cm]{\hspace*{#1}}

\bibliographystyle{IEEEtran}
%\bibliographystyle{ieeetr}


\providecommand{\mbf}{\mathbf}
\providecommand{\pr}[1]{\ensuremath{\Pr\left(#1\right)}}
\providecommand{\qfunc}[1]{\ensuremath{Q\left(#1\right)}}
\providecommand{\sbrak}[1]{\ensuremath{{}\left[#1\right]}}
\providecommand{\lsbrak}[1]{\ensuremath{{}\left[#1\right.}}
\providecommand{\rsbrak}[1]{\ensuremath{{}\left.#1\right]}}
\providecommand{\brak}[1]{\ensuremath{\left(#1\right)}}
\providecommand{\lbrak}[1]{\ensuremath{\left(#1\right.}}
\providecommand{\rbrak}[1]{\ensuremath{\left.#1\right)}}
\providecommand{\cbrak}[1]{\ensuremath{\left\{#1\right\}}}
\providecommand{\lcbrak}[1]{\ensuremath{\left\{#1\right.}}
\providecommand{\rcbrak}[1]{\ensuremath{\left.#1\right\}}}
\theoremstyle{remark}
\newtheorem{rem}{Remark}
\newcommand{\sgn}{\mathop{\mathrm{sgn}}}
\providecommand{\abs}[1]{\left\vert#1\right\vert}
\providecommand{\res}[1]{\Res\displaylimits_{#1}} 
\providecommand{\norm}[1]{\left\lVert#1\right\rVert}
%\providecommand{\norm}[1]{\lVert#1\rVert}
\providecommand{\mtx}[1]{\mathbf{#1}}
\providecommand{\mean}[1]{E\left[ #1 \right]}
\providecommand{\fourier}{\overset{\mathcal{F}}{ \rightleftharpoons}}
%\providecommand{\hilbert}{\overset{\mathcal{H}}{ \rightleftharpoons}}
\providecommand{\system}{\overset{\mathcal{H}}{ \longleftrightarrow}}
	%\newcommand{\solution}[2]{\textbf{Solution:}{#1}}
\newcommand{\solution}{\noindent \textbf{Solution: }}
\newcommand{\cosec}{\,\text{cosec}\,}
\providecommand{\dec}[2]{\ensuremath{\overset{#1}{\underset{#2}{\gtrless}}}}
\newcommand{\myvec}[1]{\ensuremath{\begin{pmatrix}#1\end{pmatrix}}}
\newcommand{\mydet}[1]{\ensuremath{\begin{vmatrix}#1\end{vmatrix}}}

\let\vec\mathbf

\vspace{3cm}

\title{
\textbf {Assignment 1}\\ \large \textbf{AI1110}: Probability and Random Variables\\Indian Institute of Techonology Hyderabad
}
\author{Ashmeet Sidhu\\BT22BTECH11004}
	


% make the title area
\maketitle

\newpage

%\tableofcontents

\bigskip

\renewcommand{\thefigure}{\theenumi}
\renewcommand{\thetable}{\theenumi}

\textbf{12.13.6.13: Question}. A bag consists of 10 balls

 each marked with one of the digits 0 to 9. If

 four balls are drawn successively with

 replacement from the bag, what is the

 probability that none is marked with 

the digit 0?\\

\textbf{Answer: $\left(\frac{9}{10}\right)^4$.}\\

\textbf{Solution}:\\

\begin{tabular}{|c|c|}
\hline
Variable & Definition \\
\hline
X & Number of balls marked with the digit 0 among the 4 balls drawn \\
\hline
p & Probability of drawing a ball marked 0 each time \\
\hline
\end{tabular}\\


%X : number of balls marked with the 

 %   digit 0 among the 4 balls drawn.\\

%p : Probability of drawing a ball 

 %   marked 0 each time. \\

Here, we need that none of the balls are marked 

0, i.e. $X=0$\\

By binomial mass disribution function :\\

\begin{tabular}{|c|c|}

\hline
n & number of trials \\
\hline
p & probability of getting 0 in each trial \\
\hline
\end{tabular}\\

%n : number of trials

%p : probability of getting 0 in each trial\\
$$ n = 4 $$
$$ p = \frac{1}{10} $$

$$ P(X=0) = \Comb{n}{0} \times (p)^0 \times (1-p)^4 $$

$$ P(X=0) = \Comb{4}{0} \times \left(\frac{1}{10}\right)^0 \times \left(\frac{9}{10}\right)^4 $$

$$ P(X=0) = 1 \times 1\times \left(\frac{9}{10}\right)^4 $$

$$ P(X=0) = \left(\frac{9}{10}\right)^4 $$

\end{document}

